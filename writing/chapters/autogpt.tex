\documentclass[../main.tex]{subfiles}

\begin{document}
\chapter{An Analysis of AutoGPT}
\label{ch:autogpt}

The general notion of an agent in computer science was introduced in \autoref{sec:agents}.
\gls{autogpt} is an open-source project that tries to 'make \gls{gpt} fully autonomous'.
The project quickly gained a lot of traction
following its first release.
The idea of an LLM that controls an agent sounded like the next step towards
real intelligence of computer systems.
Lots of programmers joined the \gls{autogpt} Repository,
and it grew into a much bigger and now-funded project.
Initially, it only contained the \gls{autogpt} agent
but over time has seen multiple additions.

The documentation of \gls{autogpt} is sparse,
and the project is developed quickly,
therefore existing documentation
can often be outdated.
Therefore, I will give a brief introduction to \gls{autogpt} and its core components
in this chapter.

What grabbed the attention of many people is not the whole project,
but merely the default agent of \gls{autogpt}.
The project further includes \emph{Forge},
an agent framework that can be used to build a custom agent.
It defines a minimal agent without logic
that abstracts away all the boilerplate.
In \autoref{sec:forge_agent} the Forge agent is introduced
and in \autoref{ch:agent} I use the Forge agent
to build a custom agent for \gls{ir}.
While the first iteration of the default agent lived in the terminal,
the project now ships with a web application
that allows interaction with agents in the browser.
Finally, a benchmarking system was introduced.
The benchmarking can be used to build level-based test suites,
to test the capabilities of an agent.
The benchmarking system is introduced in \autoref{sec:benchmarking}.

\section{Project Overview}

Originally,
the default only ran in the computer terminal
and was controlled through the command line.
Later, an option to start a server that implements the agent protocol was added.
The user can interact with the agent through a web interface in the browser.

The web interface lists all conversations an agent had and provides a chat interface for the selected one.
When the user sends an input to the agent, a task or step request is made to the server running the agent.
Furthermore, the web interface can be used to start benchmarks for an agent.
A single test or a complete test suite consisting of stages can be started.
The GUI has no support for uploading documents to an agent.
Because of this, documents for \gls{ir} would need to be placed in specific folder locations
such that the agent can access them before it is started.

\gls{autogpt} agents use workspaces in which they can act.
Each agent has its workspace and can not modify files outside it by default.
In case the user wants the agent to have access to documents, he needs to place them into its workspace by hand.
Often, the workspace is also used to save intermediate information to text files.
Some abilities produce large outputs,
and the limited context window of large language models can be exceeded quickly
if all intermediate information is appended to the prompt.

\section{Main Components}
\label{sec:default_agent}

\subsection{Default Agent}

The default agent made up the whole project when \gls{autogpt} was first released.
The official documentation \autocite{zotero-189} describes the default agent as:
\begin{quote}
    ``A \emph{generalist} agent, meaning it is not designed with a specific task in mind.
    Instead, it is designed to be able to execute a wide range of tasks across many disciplines,
    as long as it can be done on a computer.''
\end{quote}
This generalist design can be seen in the list of available abilities.
\begin{description}
    \item[Code Execution] The agent can execute arbitrary Python code given as a
        string or a file.
        The execution is contained in a docker container
        that can access the agent workspace.
    \item[File Operations] To access files in the agent workspace,
        it has abilities to list, read and write files.
    \item[Image Generation]
        The agent can make calls to different image-generation services
        that transform a prompt into an image.
    \item[Web Search]
        Web search is the most common application of the default agent.
        The agent has abilities to query a search engine
        and scrape website content using a headless browser.
    \item[Other]
        Further abilities are present in the default agent.
        The user can be prompted to give more information about the task,
        the system time can be retrieved, and a completion ability is implemented.
\end{description}
Since the development of the default agent started,
different abilities have been present and were then removed again.
For example, efforts were made to introduce long-term memory to the agent.
There were abilities to save agent conversations into a database for later retrieval.
The idea was to improve the agent's performance by using past conversations as context.
However, experiments showed that using past conversations rather hindered its capabilities. % TODO: source
This is only an example of the constant development and changing constraints
one has to deal with when working with the default agent.

\subsection{Agent Structure}

The \gls{autogpt} agent is modeled after the classic agent architecture.
After the start,
the user is asked to enter a task the \gls{autogpt} agent should perform.
Then the agent enters a loop of prompting the LLM,
executing the proposed action
handling the result of the action and updating the agent state.

The LLM is prompted in a structured way.
A base prompt template is defined
and populated with current information before each prompting step.
The information includes the task at hand,
a list of possible actions,
a history of previous actions and their results
and some extra statements that are there to guide the language model.
As the answer needs to be parsed,
the system prompt defines a fixed format the LLM should answer in.
The answer consists of the thoughts and the proposed next action.

\gls{autogpt} is divided into four modules.
The \emph{brain} is the main module that controls the agent.
In \gls{autogpt} this is realized by prompting the language model in a structured way.
Using the chat system prompt,
the language model is prompted to answer in a structured format.
Different techniques are implemented in this structure.
The language model is forced not only to plan the next step
but also to explain the choice for the chosen step and to add self-criticism.
An extra output for the human user is also returned.
The second part of the answer is the actual next action with the needed arguments.
The action makes up the second module of the agent.
In this module, the abilities of the agent are defined.
These can be file operations,
database queries or web search functionalities.
The third module is the \emph{memory}.
Memories are modeled after humans which have short and long-term memory.
Short-term memory can be implemented
as an in-memory list of messages to the language model.
Long-term memory needs persistent storage such as a database.
A popular option for language models is vector databases that work with embeddings.

Currently,
the \gls{autogpt} agent is implemented for OpenAI \gls{gpt} models.
The prompts are tailored in ways
that benefit the characteristics of GPT-3 and GPT-4.
Switching to a different model is not difficult from a software perspective,
but semantically poses a challenge.
If one knows the message format for a specific model the input can be adjusted.
However, the same prompting techniques do not necessarily work for all language models.
The challenge is to switch between different prompting styles,
as every model needs to be prompted differently.

\subsection{Forge Agent}
\label{subsec:forge_agent}

\subsection{Benchmarking System}
\label{subsec:benchmarking}

To evaluate \gls{autogpt} and other agent systems that implement the agent protocol,
the \gls{autogpt} project has implemented a benchmarking system.
The system consists of a set of tasks that the agent has to complete.
The tasks are designed to test different aspects of the agent and are divided into different topics.
Some tasks depend on the previous successful completion of other tasks.
A task consists of an input prompt and an expected output.
The output is defined by certain words that should be contained.

% - built for the hackathon
% - external agents can be tested
% - weak documentation
% - stale development

\section{Analysis for Information Retrieval}

The \gls{autogpt} Agent has different abilities that can be utilized for \gls{ir}.
It can search the web and operate on files, execute code and do other stuff. % TODO

The web search is implemented by a two-step process.
First, a search API like DuckDuckGo is called to get a list of relevant pages.
Then the page contents are scraped with a headless browser.
It is possible to read and write text files.
Other document types are processed by basic text extraction tools to get the plain text.

For longer files such as scientific journals the extracted text is too long for the language model.
The \gls{autogpt} agent cannot chunk the text into smaller chunks or store it in a database.
This is a limitation that needs to be addressed for \gls{ir} tasks over a local database repository.

Having a vector database would enable techniques such as \gls{rag}.
The agent would get a prompt with a question over the RDR and choose an action to start a semantic search over the vector database.
The result of the search is the chunks that are semantically closest to the question.
These chunks can then be included as context for the \gls{llm} prompt to generate an answer.

The default agent tends to search the web for information. We want an agent that prioritizes information that is present in the research repository.
This needs to be addressed in the prompting techniques of the agent.


\end{document}
