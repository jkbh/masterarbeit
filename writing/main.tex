\documentclass[english, version-2022-01]{uzl-thesis}
\UzLStyle{computer modern oldschool design}
\UzLThesisSetup{
Masterarbeit,
Verfasst = {am}{Institut für Informationssysteme},
Titel auf Deutsch = {Hallo Welt},
Titel auf Englisch = {Hello World},
Autor = {Jakob Horbank},
Betreuerin = {Prof. Dr. Ralf Möller},
Studiengang = {Informatik},
Datum = {15. April 2024},
Abstract = {It is about saying ``hello'' to the world.},
Zusammenfassung = {Es geht darum, der Welt »Hallo« zu sagen.},
Numerische Bibliographie
}
\begin{document}
\chapter{Introduction}
\section{Contributions of this Thesis}
This thesis says ``Hello World!'', see also \cite{Kernighan1974}.
\section{Related Work}
There are many hello world programs.

\subsection{Perplexity AI (LLM Chatbots that give sources?)}

In a variety of application contexts, the answers of an assistent have to be correct. This is espciacally true in reasearch contexts. A model that hallucinates isn't feasable in this case. But while hallucination can be reducing with fine-tuning, it can not be competely eliminated. Perplexity AI is an online service that leverages a language model to provide a search that generates an answer from different sources in the internet. The content of the answer is then linked to the found sources, so the user can see and verify the result.

As this is a propierty closed source product accessible through a web interface, this is not a useful to create our own research assistant.

\section{Structure of this Thesis}
I do this then that and then that.

In Chapter~\vref{chapter-main}, we say hello.

\chapter{Background}

\subsection{Transformer}

Until 2017, the dominating strategy to train models for language tasks revolved around recurrent structures. Every sentence token was represented as a hidden state that resultet from all the previous tokens. While this approach has a reasonable motivation, the sequential nature contraints computation speed. There is no way to compute the recurrent architectures in parallel.

The transformer model propsed by (source) solely relys on these attention mechanisms. In particular, they employ self-attention and multi-headed self-attention layers

Attention mechanism allow learning dependencies between tokens in sentences without regard to their distance. Their non-sequential nature allow for massive parallelization.

Self-Attention is an attention mechanism that computes relations between different positions of the same sequence with goal of finding a representation of the sequence.

Like a typical sequence modeling architecture, a transformer consists of an enconder and a decoder. The encoder has two parts that are stacked on each other multiple times. The first part is a mulit-head self-attention block, the second part is a classic fully connected feed foward block.

\subsection{InstructGPT}

Capturing the intent of the user is a key challenge for language models. This process is called \textit{alignment}. Even though LLMs are trained on huge datasets, they are not tailored towards human users by default. A popular approach to the alignment problem is reinforcement learning with human feedback (RLHF). Handcrafted prompts are used to fine-tune GPT-3. The outputs of the model are collected into a set and ranked by humans. This set is then used to train a reward model. With this reward model, the language model is further fine-tuned. The resulting model is called \textit{InstructGPT} and performs better than the baseline GPT-3 model.

Large language models are trained to predict the next token of a sequence, not to follow the instruction of the user. This leads to some unwanted results, such as toxic, harmful answers or fabricated information that is not true.

\subsection{AutoGPT}


\chapter{Handcrafted Prompts as a Research Chatbot}

\label{chapter-main}
\begin{itemize}
	\item First, only crafting prompt for default GPT-3.5
	\item What could be questions about hadith corpus?
	      \begin{itemize}
		      \item Ask for classifaction outputs

		            These would be the same as the BERT classifactions, just as question sentences. Questions about contained persons, locations, dates, \dots.
		      \item Further questions

		            Information probably not contained in corpus. LLM might be able to give some reasoning learned in pre-training.
	      \end{itemize}
	\item How to design prompts?

	      Few-shot with examples or Zero Shot? Include example labelings from corpus.
	\item How to include corpus?

	      The corpus can be large. Large than the context length of LLMs. So first try handling a single document
	\item How to craft testing examples?

	      For the labeling questions create examples for amended data.
\end{itemize}
\chapter{Custom Fine-Tuned Language Model as a Research Chatbot}
\begin{itemize}
	\item Here, explore fine-tuning a LLM
	\item Identify problems of non fine-tuned Chatbot
	\item How to create fine-tuned dataset

	      Again, for labeling questions from amended data, but with focus on weaknesses.
\end{itemize}
\chapter{Conclusion}
Saying hello world is quite easy.
\begin{bibtex-entries}
@TechReport{Kernighan1974,
	author = {Brian Kernighan},
	title = {Programming in C – A Tutorial},
	institution = {Bell Laboratories},
	year = {1974}
}
\end{bibtex-entries}
\end{document}
