\documentclass[english, version-2022-01]{uzl-thesis}

\usepackage{import}
\usepackage{xifthen}
\usepackage{pdfpages}
\usepackage{transparent}
\usepackage{subfiles}
\usepackage{varwidth}
\usepackage[obeyFinal, textsize=small, textwidth=3cm]{todonotes}

\usetikzlibrary{fit, positioning, external, backgrounds}
\tikzexternalize
\tikzset{thesis outline shapes}

\setuptodonotes{color=orange!50, inlinewidth=7cm}
\makeatletter
\renewcommand{\todo}[2][]{\tikzexternaldisable\@todo[#1]{#2}\tikzexternalenable}
\makeatother
\newcommand{\missingref}[1]{\todo[options]{color=red!50}{Missing references!}}

\UzLStyle{alegrya modern design}
\UzLThesisSetup{
Masterarbeit,
Verfasst = {am}{Institut für Informationssysteme},
Titel auf Deutsch = {Nutzung von AutoGPT für Informations-Recherche Agenten},
Titel auf Englisch = {Using AutoGPT for Information Retrieval Agents},
Autor = {Jakob Horbank},
Betreuerin = {Prof. Dr. Ralf Möller},
Studiengang = {Informatik},
Datum = {15. April 2024},
Abstract = {Abstract},
Zusammenfassung = {Abstract},
Numerische Bibliographie
}
\graphicspath{include/images}
\addbibresource{bib/zotero}
\setlength{\marginparwidth}{3cm}
\begin{document}

\chapter{Introduction}
\label{chap:introduction}
\subfile{chapters/introduction}

\chapter{Backgrounds}
\label{chap:backgrounds}
\subfile{chapters/backgrounds}

\chapter{Analysis of AutoGPT for Information Retrieval Tasks}
\label{chap:autogpt}
\subfile{chapters/autogpt}

\chapter{Retrieval Augmented Generation Agent}
\label{chap:agent}
\subfile{chapters/agent}

\chapter{Benchmarking for an IR Agent}
\label{chap:benchmarks}
\subfile{chapters/benchmark}

\chapter{Results}
\label{ch:results}
\subfile{chapters/results}

\chapter{Conclusion}
\label{ch:conclusion}
\subfile{chapters/conclusion}

\end{document}