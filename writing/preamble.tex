\usepackage{subfiles}
\usepackage{pgfplots}
\usepackage[acronym]{glossaries}

% Acronyms
\robustify{\gls}
\robustify{\Gls}
\glsdisablehyper
\setacronymstyle{long-short}
\newacronym{llm}{LLM}{large language model}
\newacronym{rag}{RAG}{retrieval augmented generation}
\newacronym{ir}{IR}{information retrieval}
\newacronym{gpt}{GPT}{generative pre-trained transformer}
\newacronym{bert}{BERT}{bidirectional encoder representations from transformers}
\newacronym{ai}{AI}{artificial intelligence}
\newacronym{nlp}{NLP}{natural language processing}
\newacronym{rlhf}{RLHF}{reinforcment learning with human feedback}
\newacronym{cot}{CoT}{chain-of-though}
\newacronym{qa}{QA}{question-and-answer}
\newglossaryentry{ppai}{
    name={Perplexity AI},
    description={An online search engine that answers with sources from the web}
}
\newglossaryentry{gpt3}{
    name={GPT-3},
    description={The third version of GPT, released in 2022}
}

\newglossaryentry{gpt4}{
    name={GPT-4},
    description={The fourth version of GPT, released in 2023}
}
\newglossaryentry{autogpt}{
    name={AutoGPT},
    description={An attempt at making \gls{gpt} fully autonomous}
}
\newglossaryentry{llamaindex}{
    name={LlamaIndex},
    description={A framework to build LLM apps.}
}

\newglossaryentry{sentsplitter}{
    name={SentenceSplitter},
    description={
            A text splitter that creates chunks roughly the same specified length,
            while respected sentence boundaries.}
}
\glsresetall

% Template metadata
\UzLStyle{alegrya modern design}
\UzLThesisSetup{
Masterarbeit,
Verfasst = {am}{Institut für Informationssysteme},
Titel auf Deutsch = {Nutzung von AutoGPT für Informations-Recherche Agenten},
Titel auf Englisch = {Using AutoGPT for Information Retrieval Agents},
Autor = {Jakob Horbank},
Betreuerin = {Prof. Dr. Ralf Möller},
Mit Unterstützung von = {Thomas Asselborn},
Studiengang = {Informatik},
Datum = {15. April 2024},
Abstract = {
        \renewcommand{\baselinestretch}{1.2}\selectfont
        Recent \glspl{llm} have shown impressive capabilities
        for text generation tasks.
        Researchers have started to leverage \glspl{llm} to control agents.
        \gls{autogpt} is such an attempt at creating an autonomous goal-oriented system.
        While there is work that experiments with \gls{autogpt} for web search,
        local \gls{ir} has not been researched yet.
        This thesis presents an analysis of \gls{autogpt}
        and an \gls{autogpt} agent
        that uses \gls{rag} for \gls{ir} over local documents.
        A \gls{qa} benchmark is generated from a humanities journal to test the agent
        and examine its failure points.
        The thesis concludes,
        that while using a \gls{rag} agent for \gls{ir} is an interesting
        approach there is no inherent benefit to static \gls{ir}
        systems that use \glspl{llm}.\glsresetall
    },
Zusammenfassung = {
        \renewcommand{\baselinestretch}{1.2}\selectfont
        In den letzten Jahren haben \Glspl{llm} große Fortschritte im Bereich der
        Textgeneration ermöglicht.
        Es wurden erste Versuche unternommen,
        \Glspl{llm} zur Steuerung von Agenten zu verwenden.
        \gls{autogpt} ist so ein Versuch, ein autonomes,
        zielorientiertes System zu entwickeln.
        Während es Experimente mit \gls{autogpt} zur Recherche im Internet gibt,
        wurde die Anwendung für lokale Informationsrecherche (IR) noch nicht untersucht.
        In dieser Thesis wird eine Analyse von \gls{autogpt} und ein Agent für
        IR der mit \Gls{rag} arbeitet präsentiert.
        Zur Evaluation und Beschreiben der Probleme des Agenten,
        wird ein Frage-Antwort Benchmark über eine Geisteswissenschaftliche Fachzeitschrift
        generiert.
        Die Arbeit zeigt, dass während der Ansatz eines \gls{rag} Agent für IR
        interessant ist, es keinen Vorteil gegenüber eines statischen IR Systems
        das \glspl{llm} verwendet, gibt.\glsresetall
    },
Numerische Bibliographie
}
\addbibresource{\subfix{bib/zotero}}

% For includegraphics
\graphicspath{include/images}


% Tikz
\usetikzlibrary{fit, positioning, external, backgrounds, bending, patterns, patterns.meta}
\tikzexternalize[prefix=tikz/]
\tikzset{thesis outline shapes}
\tikzset{ability/.style={block=emph blue}}
\pgfplotsset{compat=1.18}

\renewcommand{\mkbegdispquote}[2]{\openautoquote}
\renewcommand{\mkenddispquote}[2]{\closeautoquote#1#2}

%\renewcommand{\baselinestretch}{1.2}
% TODO: remove remove marker for overfull hboxes
% Show overfull in PDF
\overfullrule=1mm