\documentclass[english, version-2022-01]{uzl-thesis}
\UzLStyle{computer modern oldschool design}
\UzLThesisSetup{
Masterarbeit,
Verfasst = {am}{Institut für Informationssysteme},
Titel auf Deutsch = {Hallo Welt},
Titel auf Englisch = {Hello World},
Autor = {Jakob Horbank},
Betreuerin = {Prof. Dr. Ralf Möller},
Studiengang = {Informatik},
Datum = {15. April 2024},
Abstract = {It is about saying ``hello'' to the world.},
Zusammenfassung = {Es geht darum, der Welt »Hallo« zu sagen.},
Numerische Bibliographie
}
\begin{document}
\chapter{Introduction}
\section{Contributions of this Thesis}
This thesis says ``Hello World!'', see also \cite{Kernighan1974}.
\section{Related Work}
There are many hello world programs.
\section{Structure of this Thesis}
In Chapter~\vref{chapter-main}, we say hello.
\chapter{Main Chapter}
\label{chapter-main}
Hello World!
\chapter{Conclusion}
Saying hello world is quite easy.
\begin{bibtex-entries}
@TechReport{Kernighan1974,
	author = {Brian Kernighan},
	title = {Programming in C – A Tutorial},
	institution = {Bell Laboratories},
	year = {1974}
}
\end{bibtex-entries}
\end{document}
